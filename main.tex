\documentclass[5pt,a4paper]{article}
\usepackage[left=1.75cm,right=1.75cm,top=2cm,bottom=2cm]{geometry}
\usepackage{amsmath, amsfonts, dsfont, stmaryrd, setspace, graphicx, subcaption, float, titlesec}

%Configuration de l'affichage des liens
%\usepackage[hidelinks]{hyperref}
\usepackage[hidelinks, colorlinks=false, urlcolor=blue]{hyperref}


\title{Fonctions holomorphes }
\raggedright
\date{}
\onehalfspacing

% Définition des compteurs et réinitialisation à chaque nouvelle section
\newcounter{propcounter}[subsection]
\newcounter{defcounter}[subsection]
\newcounter{thmcounter}[subsection]

\renewcommand{\thepropcounter}{\thesubsection.\arabic{propcounter}}
\renewcommand{\thedefcounter}{\thesubsection.\arabic{defcounter}}
\renewcommand{\thethmcounter}{\thesubsection.\arabic{thmcounter}}

% Définition des commandes
\newcommand{\prop}[1]{
    \stepcounter{propcounter}
    \hypertarget{p:\thepropcounter}{}%
    \noindent\textbf{Proposition \thepropcounter ~:} #1 \newline
}
\newcommand{\propEnum}[1]{
    \stepcounter{propcounter}
    \hypertarget{p:\thepropcounter}{}%
    \noindent\textbf{Proposition \thepropcounter ~:} #1
}
\newcommand{\defin}[1]{
    \stepcounter{defcounter}
    \hypertarget{d:\thedefcounter}{}%
    \noindent\textbf{Définition \thedefcounter ~:} #1 \newline
}
\newcommand{\definEnum}[1]{
    \stepcounter{defcounter}
    \hypertarget{d:\thedefcounter}{}%
    \noindent\textbf{Définition \thedefcounter ~:} #1
}
\newcommand{\thm}[1]{
    \stepcounter{thmcounter}
    \hypertarget{t:\thethmcounter}{}%
    \noindent\textbf{Théorème \thethmcounter ~:} #1 \newline
}
\newcommand{\thmEnum}[1]{
    \stepcounter{thmcounter}
    \hypertarget{t:\thethmcounter}{}%
    \noindent\textbf{Théorème \thethmcounter ~:} #1
}
\newcommand{\demo}[1]{
    \textbf{Démonstration :~} #1 \newline
}
\newcommand{\demoEnum}[1]{
    \textbf{Démonstration :~} #1
}
\newcommand{\rmq}[1]{
    \textbf{Remarque :~} #1 \newline
}
\newcommand{\ex}[1]{
    \textbf{Exemple :~} #1 \newline
}

\renewcommand*\contentsname{Table des matières}

\begin{document}
\maketitle
\begin{onehalfspacing}

\tableofcontents


\newpage
\section{Définitions et propriétés élémentaires}

On identifie $\mathbb{C}$ à $\mathbb{R}^2$ par l'isomorphisme naturel de $\mathbb{R}$-espace vectoriel $\iota : x + iy \mapsto (x, y)$. Si $\mathcal{U}$ est un ouvert de $\mathbb{C}$, l'ensemble $\iota(\mathcal{U})$ est un ouvert de $\mathbb{R}^2$ que l'on notera $\Tilde{\mathcal{U}}$. Si $f$ est une application de $\mathbb{C} \rightarrow \mathbb{C}$, on notera $P_f$ (resp. $Q_f$) l'application $(x, y) \mapsto \textup{Re}(f(x + iy))$ (resp. $(x, y) \mapsto \textup{Im}(f(x + iy))$) et $\Tilde{f}$ l'application $(x, y) \mapsto (P_f(x, y),~ Q_f(x, y))$. Ainsi, pour tout $z = x + iy \in \mathbb{C},~ f(z) = P_f(x, y) + i Q_f(x, y)$.\newline

\defin{Soient $\mathcal{U}$ un ouvert de $\mathbb{C}$ et $f : \mathcal{U} \rightarrow \mathbb{C}$. Soit $z_0 \in \mathcal{U}$. $f$ est $\mathbb{C}$\textbf{-dérivable} en $z_0$ s'il existe $\alpha \in \mathbb{C}$ tel que 
\[ \lim_{h \to 0} \frac{f(z_0 + h) - f(z_0)}{h} = \alpha\]
}

\defin{$f : \mathcal{U} \rightarrow \mathbb{C}$ est \textbf{holomorphe} sur $\mathcal{U}$ si elle est $\mathbb{C}$-dérivable en tout point de $\mathcal{U}$. On note $f' : \mathcal{U} \rightarrow \mathbb{C}$ la fonction dérivée.}

\thm{\textit{(Admis)} Si $f : \mathcal{U} \rightarrow \mathbb{C}$ est holomorphe, alors $f'$ est continue.}

\rmq{$f$ est holomorphe sur $\mathcal{U}$ si et seulement si elle admet un \textbf{développement limité} à l'ordre $1$ en tout point de $\mathcal{U}$ : 
\[ \forall x \in \mathcal{U},~ f(x + h) = f(x) + f'(x)h + o(h) \]
Il est alors clair que $f$ est continue sur $\mathcal{U}$.
}

\prop{\textit{(Équations de Cauchy-Riemann)} Soit $f : \mathcal{U} \rightarrow \mathbb{C}$. Alors $f$ est $\mathbb{C}$-dérivable en $z_0 = x_0 + iy_0$ si et seulement si $P_f$ et $Q_f$ sont différentiables en $(x_0, y_0)$ et 

\[
    \left\{\begin{aligned}
        \frac{\partial P_f}{\partial x}(x_0, y_0) = \frac{\partial Q_f}{\partial y}(x_0, y_0) \\
        \frac{\partial P_f}{\partial y}(x_0, y_0) = \frac{-\partial Q_f}{\partial x}(x_0, y_0)
    \end{aligned}\right.
\]
}
\demo{On suppose d'abord que $f$ est $\mathbb{C}$-dérivable en $z_0$ et on note $f'(z_0) = \alpha = a + ib$. Alors, 

\begin{align*}
    f(z_0 + h) &= f(z_0) + \alpha \times h + o(h) \\
    f(z_0 + h) &= f(z_0) + (a + ib) \times (h_x + ih_y) + o(h) \\
    f(z_0 + h) &= f(z_0) + (ah_x - bh_y) + i(ah_y + bh_x) + o(h)
\end{align*}

donc, 
\[
    \tilde{f}(x_0 + h_x,~ y_0 + h_y) = \tilde{f}(x_0,~ y_0) + (ah_x - bh_y,~ ah_y + bh_x) + o(h_x, h_y)
\]
que l'on peut ré-écrire 
\[
    \tilde{f}(x_0 + h_x,~ y_0 + h_y) = \tilde{f}(x_0,~ y_0) + 
        \begin{pmatrix}
            a & -b \\
            b & a
        \end{pmatrix}
        \begin{pmatrix}
            h_x \\
            h_y
        \end{pmatrix}
    + o(h_x, h_y)
\]
On en déduit que $\tilde{f} = (P_f, Q_f)$ est différentiable en $(x_0, y_0)$ et que sa jacobienne est la matrice 
\[
 \textup{Jac}_{(x_0, y_0)} \tilde{f} := 
    \begin{pmatrix}
        \frac{\partial P_f}{\partial x}(x_0, y_0) & \frac{\partial P_f}{\partial y}(x_0, y_0) \\
        \frac{\partial Q_f}{\partial x}(x_0, y_0) & \frac{\partial Q_f}{\partial y}(x_0, y_0)
    \end{pmatrix}
    =
    \begin{pmatrix}
        a & -b \\
        b & a
    \end{pmatrix}
\]
Réciproquement, on suppose que $P_f$ et $Q_f$ sont différentiables en $(x_0, y_0)$ et que
\[
    \left\{\begin{aligned}
        a := \frac{\partial P_f}{\partial x}(x_0, y_0) = \frac{\partial Q_f}{\partial y}(x_0, y_0) \\
        b := \frac{\partial P_f}{\partial y}(x_0, y_0) = \frac{-\partial Q_f}{\partial x}(x_0, y_0)
    \end{aligned}\right.
\]
Alors $\tilde{f}$ est différentiable en $(x_0, z_0)$ et 
\[  
    \textup{Jac}_{(x_0, y_0)} \tilde{f}
    =
    \begin{pmatrix}
        a & -b \\
        b & a
    \end{pmatrix}
\]
donc on retrouve le fait que $f$ est $\mathbb{C}$-dérivable en $z_0$ en écrivant un développement limité à l'ordre 1 en $(x_0, y_0)$ de $\tilde{f}$.
}

\rmq{On a de plus $f'(x_0 + iy_0) = \frac{\partial P_f}{\partial x}(x_0, y_0) - i \frac{\partial P_f}{\partial y}(x_0, y_0)$.}

\ex{La conjugaison complexe ($f : z \mapsto \Bar{z}$) n'est $\mathbb{C}$-dérivable en aucun point de $\mathbb{C}$. En effet, les fonctions $P_f(x, y) = x$ et $Q_f(x, y) = -y$ ne vérifient pas les équations de Cauchy-Riemann.}

\rmq{Les équations de Cauchy-Riemann peuvent être reformulées autrement : $f$ est $\mathbb{C}$-dérivable en $z_0 = x_0 + iy_0$ si et seulement si $\tilde{f}$ est différentiable en $(x_0, y_0)$ et sa différentielle est $\mathbb{C}$-linéaire.}

\propEnum{Soit $\mathcal{U}$ un ouvert de $\mathbb{C}$. On note $\mathcal{H}(\mathcal{U})$ l'ensemble des fonctions holomorphes sur $\mathcal{U}$. Alors,
\begin{enumerate}
    \item $\mathcal{H}(\mathcal{U})$ est un $\mathbb{C}$-espace vectoriel.
    \item Si $f, g \in \mathcal{H}(\mathcal{U})$, alors $fg \in \mathcal{H}(\mathcal{U})$ et $(fg)' = f'g + fg'$.
    \item Si $f \in \mathcal{H}(\mathcal{U})$ et si $f$ ne s'annule pas sur $\mathcal{U}$, alors $\frac{1}{f} \in \mathcal{H}(\mathcal{U})$ et $(\frac{1}{f})' = \frac{-f'}{f^2}$.
    \item Si $f : \mathcal{U} \rightarrow \mathbb{C}$ et $g : \mathcal{V} \rightarrow \mathbb{C}$ sont holomorphes et que $f(\mathcal{U}) \subset \mathcal{V}$, alors $g \circ f \in \mathcal{H}(\mathcal{U})$ et $(g \circ f)' = f' \times (g' \circ f)$.
\end{enumerate}
}
\demo{Même preuve que pour les fonctions dérivables sur un ouvert de $\mathbb{R}$.}

\ex{Les polynômes sont holomorphes sur $\mathbb{C}$, les fractions rationnelles sont holomorphes sur tout $\mathbb{C}$ sauf sur leurs pôles.}

\thm{\textit{(Théorème des accroissements finis)} Soit $E$ un espace vectoriel normé, $\mathcal{U}$ un ouvert connexe de $E$ et $f : \mathcal{U} \rightarrow \mathbb{R}$ une fonction différentiable. Alors pour tout $a, b \in \mathcal{U}$, il existe $c \in [a, b]$ tel que 
\[ f(a) - f(b) = D_cf(b-a) \]
}

\thm{\textit{(Inégalité des accroissements finis)} Soient $E, F$ deux espaces vectoriels normés, $\mathcal{U}$ un ouvert convexe de $E$ et $f : \mathcal{U} \subset E \rightarrow F$ un fonction différentiable. Soient $a, b \in \mathcal{U}$. Alors,
\[ \lVert f(a) - f(b) \rVert \leq \textup{sup}_{x \in [a, b]} \lVert D_xf \rVert \cdot \lVert (b-a) \rVert \]
}

\thm{(Rappel) Soit $\mathcal{U}$ un ouvert de $\mathbb{R}^2$ et $f : \mathcal{U} \rightarrow \mathbb{R}^m$. Si $f$ admet des dérivées partielles sur $\mathcal{U}$ et si elles sont \textbf{continues} en un point $a \in \mathcal{U}$, alors $f$ est différentiable en $a$ et $D_af (h_1, h_2) = h_1 \cdot \frac{\partial f}{\partial x}(a) + h_2 \cdot \frac{\partial f}{\partial y}(a)$.}
\demo{On note $a = (a_1, a_2)$, $D(h = (h_1, h_2)) = h_1 \cdot \frac{\partial f}{\partial x}(a) + h_2 \cdot \frac{\partial f}{\partial y}(a)$ et $u(h) = f(a + h) - f(a) - D(h)$ et on va montrer que $u(h) = o(h)$. \\
Soit $\varepsilon > 0$. Les dérivées partielles de $f$ sont continues donc il existe $\delta > 0$ tel que pour tout $z \in \textup{B}(0, \delta)$,
\[ \lVert \frac{\partial f}{\partial x~ (\textup{resp. }\partial y)}(a + z) - \frac{\partial f}{\partial x~ (\textup{resp. }\partial y)}(a)\rVert \leq \varepsilon \]

Soit $(h_1, h_2) \in B(0, \delta)$. On écrit
\begin{align*}
    u(h_1, h2) &= f(a_1 +h_1, a_2 + h_2) - f(a_1, a_2) - h_1 \cdot \frac{\partial f}{\partial x}(a) - h_2 \cdot \frac{\partial f}{\partial y}(a) \\
    &= \left[ f(a_1 + h_1, a_2 + h_2) - f(a_1, a_2 + h_2) - h_1 \cdot \frac{\partial f}{\partial x}(a) \right]
    + \left[ f(a_1, a_2 + h_2) - f(a_1, a_2) - h_2 \cdot \frac{\partial f}{\partial y}(a) \right] ~\textup{\textbf{(1)}}
\end{align*}
et on pose 
\begin{align*}
    v(t) &= f(a_1 + t, a_2 + h_2) - f(a_1, h_2 + h_2) - t \cdot \frac{\partial f}{\partial x}(a) \\
    w(t) &= f(a_1, a_2 + t) - f(a_1, a_2) - t \cdot \frac{\partial f}{\partial y}(a)
\end{align*}
$v$ est dérivable sur $[0, h_1]$ et pour tout $t \in [0, h_1],~ v'(t) = \frac{\partial f}{\partial x}(a_1 + t, a_2 + h_2) - \frac{\partial f}{\partial x}(a)$ donc $\lVert v'(t) \rVert \leq \varepsilon$. De même, pour tout $t \in h_2,~ \lVert w'(t) \rVert \leq \varepsilon$. On déduit de l'inégalité des accroissements finis que $\lVert v(h_1) \rVert \leq |h_1|\varepsilon$ et $\lVert w(h_2) \rVert \leq |h_2|\varepsilon$. En utilisant l'expression \textbf{(1)} puis l'inégalité triangulaire, on trouve 
\[ \lVert u(h_1, h_2) \rVert \leq |h_1|\varepsilon + |h_2|\varepsilon \leq 2\varepsilon \cdot \lVert h \rVert_{\infty} \]
ce qui permet de conclure.  
}

\prop{Soient $\mathcal{U} \subset \mathbb{C}$ un ouvert \textbf{convexe} et $f : \mathcal{U} \rightarrow \mathbb{C}$ holomorphe. $f$ est constante sur $\mathcal{U}$ si et seulement si $f' = 0$.}
\demo{Si $f$ est constante, on écrit le taux d’accroissement et on a bien $f' = 0$.\\
Supposons maintenant que $f' = 0$. Alors, les applications $\frac{\partial P_f}{\partial x}$ et $\frac{\partial P_f}{\partial y}$ sont nulles, et donc continues. Comme $f$ est holomorphes, on en déduit que les dérivées partielles de $Q_f$ sont aussi nulles. Selon le théorème (\textbf{1.0.3}), $P_f$ et $Q_f$ sont différentiables et leurs différentielles sont nulles. En appliquant le théorème des accroissements finis sur $\Tilde{U}$, on trouve que $P_f$ et $Q_f$ y sont constantes, et donc que $f$ est constante sur $\mathcal{U}$.}


\newpage
\section{Rappels sur les séries entières}

\defin{Soit $\sum a_nz^n$ une série entière. \\
Le \textbf{rayon de convergence} de $\sum a_nz^n$ est $\textup{sup}(\{~ t \geq 0,~ |a_nt^n| \textup{ est bornée}~ \})$ et est noté $R_a$.}

\prop{Soit $\sum a_nz^n$ une série entière. Soit $z \in \mathbb{C}$. Si $|z| < R_a$, alors la série $\sum a_nz^n$ est absolument convergente. Si $|z| > R_a$, alors la série $\sum a_nz^n$ diverge. Si $|z| = R_a$, la série peut converger ou diverger.}

\prop{\textit{(Règle de d'Alembert)} Soit $\sum a_nz^n$ une série entière telle que la suite $(a_n)$ est non nulle à partir d'un certain rang. On suppose de la suite $\frac{|a_{n+1}|}{|a_n|}$ converge vers $l \in \overline{\mathbb{R}}$. Alors $R_a = 1/l$, avec les conventions $1/0 = +\infty$ et $\frac{1}{+\infty} = 0$.}

\prop{\textit{(Règle de Cauchy)} Soit $\sum a_nz^n$ une série entière. Si $|a_n|^{1/n}$ converge vers $l \in \overline{\mathbb{R}}$, alors $R_a = 1/l$ (avec les mêmes conventions que précédemment).}

\prop{\textit{(Règle de Cauchy améliorée)} Soit $\sum a_nz^n$ une série entière. Alors $1/R_a = \textup{limsup }a_n^{1/n}$.}

\defin{Soit $\sum a_nz^n$ une série entière. La \textbf{série dérivée} de $\sum a_nz^n$ est la série $\sum (n+1)a_{n+1}z^n$. Son rayon de convergence est noté $R_a'$.}

\prop{Soit $\sum a_nz^n$ une série entière. Alors, $R_a = R_a'$.}
\demo{Supposons que $R_a < R_a'$. Soit $z_0 \in \mathbb{C}$ tel que $R_a < |z_0| < R_a'$. Alors, 
\[ |a_{n+1}z_0^{n+1}| \leq (n+1)|a_{n+1}z_0^n||z_0|\]
La suite de droite est bornée donc celle de gauche aussi, ce qui est absurde. 
Supposons maintenant que $R_a' < R_a$. Soient $z_0 \in \mathbb{C}$ et $l > 0$ tels que $R_a' < |z_0| < l < R_a$. On note $M > 0$ un majorant de la suite $|a_nt^n|$. Alors, 
\[ |(n+1)a_{n+1}z_0^{n+1}| = (n+1)|a_{n+1}l^{n+1}|(z_0/l)^{n+1} \leq (n+1)M(z_0/l)^{n+1}\]
Par croissances comparées, on sait que la série des $(n+1)M(z_0/l)^{n+1}$ converge, donc celle des $|(n+1)a_{n+1}z^n|$ aussi, ce qui est absurde.
}


\newpage
\section{Fonctions analytiques}

\prop{Soit $\sum a_nz^n$ une série entière. On note $f : B(0, R_a) \rightarrow \mathbb{C},~ z \mapsto \sum_{n=0}^{+\infty} a_nz^n$. $f$ est holomorphe sur $B(0, R_a)$ et pour tout $z \in B(0, R_a)$, $f'(z) = \sum_{n=0}^{+\infty} (n+1)a_{n+1}z^n$.}
\demo{Soit $z \in B(0, R_a)$ et $r > 0$ tel que $|z| < r < R_a$. Pour tout $h \in \mathbb{C}$ tel que $|z| + |h| < r$, 
\begin{align*} 
    \frac{f(z + h) - f(z)}{h} - \sum_{n = 0}^{+\infty} (n+1)a_{n+1}z^n 
    &= \frac{1}{h} \sum_{n = 0}^{+\infty} a_n \left[ (z + h)^n - z^n \right] - \sum_{n = 0}^{+\infty} (n+1)a_{n+1}z^n \\
    &= \sum_{n=1}^{+\infty} a_n \left[ \frac{(z+h)^n - z^n}{h} - nz^{n-1} \right]
\end{align*}
Pour tout $n \geq 1$, on pose 
\[ v_n(h) = \frac{(z+h)^n - z^n}{h} - nz^{n-1} \]
Il est clair que $\lim_{h \to 0}~ v_n(h) = 0$. De plus,
\begin{align*}
    |v_n(h)| &= |\frac{(z+h)^n - z^n}{h} - nz^{n-1}| \\
    &= |\sum_{k=0}^{n-1} (z+h)^kz^{n-k} - nz^{n-1}| \\
    &\leq \sum_{k = 0}^{n-1} |z+h|^k|z|^{n-k} + n|z|^{n-1} \\
    &\leq 2nr^{n-1}
\end{align*}
car $|z| \leq |z| + |h| < r$. On en déduit que la série $\sum a_nv_n(h)$ est converge normalement. On peut donc passer à la limite dans la série :
\begin{align*} 
    \lim_{h \to 0}~ \sum_{n=1}^{+\infty} a_n \left[ \frac{(z+h)^n - z^n}{h} - nz^{n-1} \right] 
    &= \sum_{n=1}^{+\infty} a_n \cdot \lim_{h \to 0}~ \left[ \frac{(z+h)^n - z^n}{h} - nz^{n-1} \right] \\
    &= 0
\end{align*}
}

\rmq{On en déduit que $f$ est infiniment $\mathbb{C}$-dérivable sur $B(0, R_a)$ et que pour tout $n \in \mathbb{N}$, $a_n = f^{(n)}(0)/n!$. On peut alors écrire le \textbf{développement en série de Taylor en 0} de $f$ : 
\[ f(z) = \sum_{n=0}^{+\infty} \frac{f^{(n)}(0)}{n!}z^n \]
}

\defin{Soit $\mathcal{U}$ un ouvert de $\mathbb{C}$. Une fonction $f : \mathcal{U} \rightarrow \mathbb{C}$ est \textbf{analytique} lorsque pour tout $z_0 \in \mathcal{U}$, il existe $B(z_0, R) \subset \mathcal{U}$ et une série entière $\sum a_nz^n$ telle que $R \leq R_a$ et $f(z) = \sum_{n=0}^{+\infty}(z - z_0)^n$ pour tout $z \in B(0, R)$.}

\prop{Soit $f : \mathcal{U} \rightarrow \mathbb{C}$ analytique. Alors $f$ est holomorphe sur $\mathcal{U}$ et admet des dérivées de tous ordres qui sont toutes holomorphes. }
\demo{En effet, si $z_0 \in \mathcal{U}$, il existe une série entière $\sum a_nz^n$ telle que pour tout $z \in B(0, R_a) \cap \mathcal{U}$,
\begin{align*} 
    f(z) &= \sum_{n=0}^{+\infty}a_n(z - z_0)^n \\
    &= \left[ (h \mapsto \sum_{z=0}^{+\infty}a_nh^n) \circ (u \mapsto u - z_0) \right](z)
\end{align*}
Ces deux fonctions sont holomorphes donc leur composition l'est aussi.
}

\prop{Soit $\sum a_nz^n$ une série entière. Alors la fonction $f : B(0, R_a) \rightarrow \mathbb{C}$ associée à cette série est analytique.}
\demo{Soit $z_0 \in B(0, R_a)$. On veut montrer qu'il existe une série entière $\sum b_nz^n$ telle que pour tout $z \in B(z_0, R_b) \cap B(0, R_a)$, on ait
\[ f(z) = \sum b_n(z - z_0)^n\]
Soit $p \in \mathbb{N}$. 
\begin{align*}
    f^{(p)}(z_0) &= \sum_{n = p}^{+\infty} n(n-1)...(n-p+1)a_nz_0^{n-p} \\
    &= \sum_{n = p}^{+\infty} \frac{n!}{(n-p)!}a_nz_0^{n-p} \\
    &= \sum_{q=0}^{+\infty} \frac{(p+q)!}{q!}a_{p+q}z_0^q ~~(\textup{changement de variable } q = n - p) \\
    &= \sum_{q=0}^{+\infty} p!\binom{p+q}{q}a_{p+q}z_0^q
\end{align*}
On en déduit que la série entière $\sum (f^{(p)}(z_0)/p!) z^n$ a un rayon de convergence supérieur à $R_a - |z_0|$. En effet, si $r < R_a - |z_0|$, alors 
\begin{align*}
    |\frac{f^{(p)}(z_0)}{p!}r^p| &\leq \sum_{q=0}^{+\infty}\binom{p+q}{q}|a_{p+q}| \cdot |z_0|^qr^p \\
    &\leq \sum_{n=p}^{+\infty}\binom{n}{n-p}|a_{n}| \cdot \textup{max}(|z_0|, r)^n ~~(\textup{changement de variable } n = p + q)\\
    &< +\infty
\end{align*}
car $\textup{max}(|z_0|, r) < R_a$. Soit $z \in \mathbb{C}$ tel que $|z - z_0| < R_a - |z_0|$. Alors,
\begin{align*}
    \sum_{p=0}^{+\infty} \frac{f^{(p)}(z_0)}{p!}(z - z_0)^n &= \sum_{p = 0}^{+\infty} \sum_{q = 0}^{+\infty} \binom{p+q}{q}a_{p+q}z_0^q(z-z_0)^p \\
    &= \sum_{n=0}^{+\infty} a_n \sum_{p+q=n} \binom{n}{q}z_0^q(z-z_0)^{n-q} ~~\textup{\textbf{(1)}} \\
    &= \sum_{n=0}^{+\infty} a_n z^n \\
    &= f(z)
\end{align*}
\textbf{(1)} vient du fait que la série est absolument convergente donc on peut la réordonner comme on veut : $\mathbb{N}^2 = \cup_{n = 0}^{+\infty}~ \{(p,q) \in \mathbb{N}^2,~ p+q=n\}$
}

\end{onehalfspacing}
\end{document}